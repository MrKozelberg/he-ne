\documentclass[12pt, russian, a4paper]{article}

\input{pre}

\begin{document}
	\input{titlepage}
	\setcounter{page}{2}

	\Section{ТЕОРИЯ}

	\Subsection{ЧАСТОТНЫЕ ХАРАКТЕРИСТИКИ ГАЗОВОГО HE-NE ЛАЗЕРА}

	В He-Ne лазере в качестве активной усиливающей свет среды используется смесь инертных газов гелия и неона при низком давлении (единицы миллиметров ртутного столба), которая возбуждается тлеющим разрядом. Инверсия населенности уровней создается в нейтральных атомах неона, а гелий играет вспомогательную роль «резервуара» для создания избыточно энергии, полученной при столкновении с электронами. Гелий в процессе неупругих столкновений с неоном передает ему энергию, населяя верхний рабочий уровень. Нижним возбужденным состояниям атома гелия $2^1s_0$ и $2^3s_1$ соответствуют избыточная внутренняя энергия 20,61 и 19,82 эВ соответственно.

	He-Ne лазер может работать в двух режимах генерации: в режиме  генерации на нескольких продольных модах резонатора и в режиме одночастотной генерации. В первом случае спектр состоит из чрезвычайно узких спектральных линий, во втором случае --- из одной спектральной линии. Как следует из рисунка \ref{fig:he-ne-enrgy} генерация на волне 632.8 нм обеспечивается переходом $3s_2\rightarrow2p_4$ в атоме Ne, генерация на волне 1153 нм --- переходом $2s_2\rightarrow2p_4$, а генерация на волне 3309 нм --- переходом $3s_2\rightarrow3p_4$.

	\begin{figure}[htbp]
		\centering
		\includegraphics[scale=0.8]{../images/he-ne-enrgy.png}
		\caption{Диаграмма нижних энергетических состояний гелия и неона.}
		\label{fig:he-ne-enrgy}
	\end{figure}

	Переходы $3s_2\rightarrow2p_4$, $2s_2\rightarrow2p_4$ и $3s_2\rightarrow3p_4$ имеют различный характер уширения. Характер уширения однородный или не однородный, когда в процессе генерации на одной частоте участвуют, соответственно, все, или часть компонент спектральной линии, определяется соотношением трех процессов. Это естественное, столкновительное и доплеровское уширения. В He-Ne лазере при генерации на переходе $3s_2\rightarrow2p_4$ ($\lambda = 632.8\,\textnormal{нм}$) и на переходе $2s_2\rightarrow2p_4$ ($\lambda = 1153\,\textnormal{нм}$) уширение спектральной линии имеет неоднородный характер, в то время как при генерации $3s_2\rightarrow3p_4$ ($\lambda = 3309\,\textnormal{нм}$).

	Рассмотрим основные особенности спектра излучения, формируемого при помощи резонатора Фабри-Перо в пределах части спектральной линии спонтанного излучения, в которой усиление света превышает уровень потерь. Под \emph{полной шириной полосы резонатора на полувысоте} понимают
	\begin{equation}
		\Delta \nu_c = \dfrac{cf}{2\pi L},\quad f = \mathrm{ln}\qty(\dfrac{1}{\sqrt{R_1R_2}}),
	\end{equation}
	где $c$ --- скорость света в вакууме, $R_{1,2}$ --- коэффициенты отражения зеркал, $f$ --- потери за один обход. Величина $c/2L$ определяет частотный интервал между основными типами колебаний интерферометра Фабри-Перо. Заметим, что для $f=0.009$ и $L=0.4\,\textnormal{м}$ $\Delta \nu_c$ получается примерно $1\,\textnormal{МГц}$, а $c/2L$ составляет около 375 МГц.

	\emph{Однородной шириной} $\Delta \nu_a$ перехода $3s_2\rightarrow2p_4$ ($\lambda = 632.8\,\textnormal{нм}$) изолированного атома Ne является \emph{естественная ширина} линии $\Delta \nu_n$, которая определяется конечностью времени жизни верхнего и нижнего уровней энергии перехода и составляет около 20 МГц.

	Оптимальное давление газовой смеси He-Ne для генерации на волне 632.8 нм, настолько низко, что доплеровское уширение линии спонтанного излучения многократно превышает уширение из-за столкновений. \emph{Доплеровский сдвиг} можно оценить следующим образом:
	\begin{equation}
		\Delta \nu_D = 7\times10^{-7}\,\nu_0\,\sqrt{\dfrac{T \textnormal{ [К]}}{M}},
	\end{equation}
	где $M=20$ – молекулярная масса, $T \textnormal{ [К]}$ – температура в Кельвинах, $\nu_0$ - частота перехода. При $T=400\,\textnormal{К}$ величина доплеровского сдвига на волне 632 нм составляет $\sim$1500 МГц. Таким образом, линия спонтанного излучения многократно уширена относительно естественной ширины линии перехода и приблизительно равна 1500 МГц. Поэтому переход $3s_2\rightarrow2p_4$ ($\lambda = 632.8\,\textnormal{нм}$) является \emph{неоднородно уширенным}, то есть в пределах данного перехода возможно возбуждение нескольких частот резонатора Фабри-Перо.

	Итог такому рассмотрению спектральных характеристик можно заключить с помощью следующего неравенства:
	\begin{equation}
		\Delta \nu_c < \Delta \nu_n = \Delta \nu_a < c/2L < \Delta \nu_D.
	\end{equation}

	При многократном прохождении света в прямом и обратном направлениях, при усилении в активной среде, превышающем потери, возбуждается генерация с пространственной структурой поля, присущей собственным модам резонатора Фабри-Перо таким, что на двойном обходе резонатора укладывается целое число длин волн. Частоты соседних резонансных мод отличаются друг от друга на $c/2L$. Эти возбужденные моды активного резонатора Фабри-Перо равномерно расположены в пределах доплеровски уширенной линии усиления. В He-Ne лазерах, как правило, в пределы ширины линии спонтанного излучения попадает более одного типа колебаний лазера. Усиливают, в основном, те атомы, доплеровски-смещенные частоты которых расположены вблизи собственных частот резонатора Фабри-Перо в интервале порядка естественной ширины линии рабочего перехода. Поэтому в неоднородно уширенном переходе при $c/2L > \Delta\nu_n$, генерация может происходить на соседних продольных модах резонатора Фабри-Перо без заметного влияния друг на друга.

	Спектральная ширина индуцированного излучения гораздо уже упомянутых выше спектральных ширин. Так называемая \emph{естественная ширина линии генерации}, величина которой обусловлена только спонтанным излучением и без учета технических возмущений, определяется \emph{теоретической шириной}:
	\begin{equation}
		\Delta\nu_T \approx \dfrac{8\pi h \nu}{P}\qty(\dfrac{\Delta\nu_c \Delta\nu_D}{\Delta\nu_c + \Delta\nu_D})^2 \approx \dfrac{8\pi h \nu \qty(\Delta \nu_c)^2}{P},
	\end{equation}
	где последнее выражение учитывает $\Delta\nu_c \ll \Delta\nu_D$. Здесь $P$ --- мощность генерации в одном типе колебаний, $\nu$ --- оптическая частота колебаний.

	На рисунке \ref{fig:freq-width} представлены схематически определенные выше характерные формы спектральной линии спонтанного перехода, спектра резонатора Фабри-Перо и спектральных линий излучения He-Ne лазера в многомодовом режиме.

	\begin{figure}[htbp]
		\centering
		\includegraphics[scale=0.8]{../images/freq-width.png}
		\caption{К определению спектра выходного излучения: 1 --- относительный уровень выходной мощности; 2 --- относительный уровень потерь. Здесь $\Delta\nu = c/2L$.}
		\label{fig:freq-width}
	\end{figure}

	Ширина спектра частот, излучаемых лазером, характеризуется понятием \emph{степени монохроматичности излучения} $\mu$, которое для спектральной линии определяется как 
	\begin{equation}
		\mu = \dfrac{\Delta\nu}{\nu} = \dfrac{\Delta\lambda}{\lambda},
	\end{equation}
	где $\Delta\nu$ --- полуширина линии, а $\nu$ --- центральная частота. Для случая протяжённого участка спектра 
	\begin{equation}
		\mu = \ln\dfrac{\nu_2}{\nu_1} = \ln{\lambda_1}{\lambda_2},
	\end{equation}
	где $\lambda_1$ и $\lambda_2$ --- границы участка спектра. Если $\mu\approx0$, то излучение монохроматично. Квазимонохроматичным излучение может быть названо, если ${\Delta\nu}/{\nu}\ll1$.

	\Subsection{ПРОВАЛ ЛЭМБА}

	В одночастотном режиме генерации газового лазера на разрешенном электронном переходе с неоднородным контуром усиления можно наблюдать интересную спектральную особенность. В области центральной частоты перехода $\nu_0$ наблюдается провал в мощности генерации, который называется \emph{Лэмбовским}. 

	Усиление в газовых лазерах происходит путём взаимодействия световой волны с молекулами активной среды --- газа. В газе молекулы двигаются, из-за чего взаимодействие происходит не на центральной частоте $\nu_0$, а на доплеровски сдвинутой частоте $(1\pm v/c)\nu_0$. Внутри лазерного резонатора стоячая структура поля образуется волнами, распространяющихся в обоих направлениях. При этом, молекула, обладающая некоторой скоростью $v$, может взаимодействовать как со встречной волной, для которой резонансный переход будет смещен в область более высоких частот, так и с догоняющей, для которой резонансный переход сдвигается к более низким частотам. В линии усиления при этом происходит снижение усиления на частотах взаимодействия в областях, так называемых \emph{Беннетовских провалов}, расположенных симметрично относительно центральной частоты $\nu_0$ (см. рис. \ref{fig:bennet-failure}). Если скорости движения молекул газа малы, то Беннетовские провалы сближаются и перекрываются, в итоге образуется один провал на центральной частоте, это и есть Лэмбовский провал.

	\begin{figure}[htbp]
		\centering
		\includegraphics[scale=0.5]{../images/bennet-failure.png}
		\caption{Бегущие волны и результирующие Беннетовские провалы в лазере со стоячей структурой поля.}
		\label{fig:bennet-failure}
	\end{figure}

	Профиль Лэмбовского провала в мощности $I(y)$ в линии генерации одночастотного лазера при строго неоднородном уширении линии описывается уравнением
	\begin{equation}\label{eq:I_y}
		I(y) = I_\text{sat}\, \dfrac{r^2 \exp\qty(-8\alpha y^2) - 1}{f(y)}.
	\end{equation}
	Здесь $r$ --- коэффициент превышения (порога генерации), описывающий отношение ненасыщенного показателя усиления в центре линии к потерям; переменная $y = 2 (\nu-\nu_0)/\Delta\nu_a$ --- отстройка частоты генерации от центра линии усиления, которая измеряется в единицах половины однородной ширины линии ($\Delta\nu_a/2)$; коэффициент $\alpha=(\ln(2) / 4)(\Delta\nu_a / \Delta\nu_d)^2$ описывает отношение однородной ширины к доплеровской ширине ($\Delta\nu_d$); фактор $f(y) \approx (2+y^2)/ (1+y^2)$ изменятся от 2 при малом аргументе до 1 при большом значении аргумента; величина $I_\text{sat}$ описывает мощность насыщения коэффициента усиления вблизи центра линии двумя встречными волнами согласно следующему закону:
	\begin{equation}
		\alpha_m \approx \dfrac{\alpha_{m0}}{N},
	\end{equation}
	где $N = \sqrt{1 + 2I(0)/I_\text{sat}}$ --- параметр насыщения коэффициента усиления в центре линии генерации (для неоднородно уширенной линии усиления).

	Мощность на выходе из лазера пропорциональна генерируемой мощности. Это позволяет по форме Лэмбовского провала в спектральной форме выходной мощности, задавшись известным значением доплеровской ширины $\Delta\nu_d$, определять на основе соотношения (\ref{eq:I_y}) однородную ширину $\Delta\nu_a$, коэффициент превышения над порогом $r$ и параметр насыщения $N$.

	\Section{ПРАКТИЧЕСКИЕ ЗАДАНИЯ}

	\Subsection{ОПИСАНИЕ УСТАНОВКИ}

	Установка (см. рис. \ref{fig:scheme}) состоит из двух гелий-неоновых лазеров ОКГ-13 и ЛГН 223-1, соединённых через волоконный ответвитель с трёхзеркальным кольцевым резонатором. В оптический  уть лучей каждого лазера встроены волоконные контроллеры поляризации ПУ1, ПУ2, выполняющие роль пластинок $\lambda/2$ и $\lambda/4$. Выход кольцевого резонатора направлен на фотодетектор, сигнал с которого подаётся на осциллограф. С помощью зеркал $\text{З}_3\text{-З}_7$ можно создать обратную связь в лазере ЛГН 223-1.

	\begin{figure}[htbp]
		\centering
		\includegraphics[width=\textwidth]{../images/scheme.png}
		\caption{Принципиальная схема установки.}
		\label{fig:scheme}
	\end{figure}


	\Subsection{НАБЛЮДЕНИЕ СПЕКТРА ИЗЛУЧЕНИЯ HE-NE ЛАЗЕРА С РЕЖИМОМ ГЕНЕРАЦИИ ОДНОЙ И БОЛЕЕ МОД ПРИ ПОМОЩИ ТРЁХЗЕРКАЛЬНОГО КОЛЬЦЕВОГО РЕЗОНАТОРА, НАБЛЮДЕНИЕ ЗАВИСИМОСТИ МОЩНОСТИ ГЕНЕРАЦИИ ОТ ЧАСТОТЫ НА ВЫХОДЕ ЛАЗЕРА ОКГ-13}

	Перед началом работы мы убедились, что в оптоволокно попадает максимально возможная мощность, а резонатор настроен максимально точно. Затем при помощи поляроидной пленки и настройки контроллеров поляризации получили максимальный отклик от вертикальной поляризации излучения. Вид полученного спектра представлен на рисунке \ref{fig:1}.

	\begin{figure}[htbp]
		\centering
		\includegraphics[width=0.8\textwidth]{../images/1.PNG}
		\caption{Спектр ОКГ-13 при напряжении $U=1.581\,\text{В}$.}
		\label{fig:1}
	\end{figure}

	От преподавателя мы узнали, что длина резонатора лазера ОКГ-13 составляет $L = 22.5\,\text{см}$, на основе чего было рассчитано межмодовое расстояние для данного лазера $c/2L = 666\,\text{МГц}$.

	На рисунке \ref{fig:1} расстояние между двумя большими пиками равно межмодовому, отсюда не трудно вычислить ширину пиков на их полувысоте (для этого были использованы измерения линейкой изображения, выведенного на монитор компьютера). В результате измерений было установлено, что $\Delta \nu = 44.2 \pm 7.9\,\text{МГц}$, где погрешность берётся из-за конечной толщины луча осциллографа.

	При перестройке частоты модуляции периметра резонатора наблюдали конкурентный характер смены генерируемых мод (см. рис. \ref{fig:0.2}).

	\begin{figure} 
    	\centering
    	\begin{subfigure}[tb]{.32\textwidth}
			\centering
		 	\includegraphics[width=\textwidth]{../images/1.PNG}
			\caption{}
			\label{fig:0.2-1}
		\end{subfigure}
    	\hfill
    	\begin{subfigure}[tb]{.32\textwidth}
			\centering
			\includegraphics[width=\textwidth]{../images/2.PNG}
			\caption{}
			\label{fig:0.2-2}
		\end{subfigure}
    	\hfill
    	\begin{subfigure}[tb]{.32\textwidth}
			\centering
			\includegraphics[width=\textwidth]{../images/3.PNG}
			\caption{}
			\label{fig:0.2-3}
		\end{subfigure}
     	\caption{Спектры лазера ОКГ-13 при различных частотах модуляции периметра. Масштабный фактор 0.2.}
     	\label{fig:0.2}
	\end{figure}

	\begin{figure} 
    	\centering
    	\begin{subfigure}[tb]{.49\textwidth}
			\centering
		 	\includegraphics[width=\textwidth]{../images/4.PNG}
			\caption{}
			\label{fig:0.05-4}
		\end{subfigure}
    	\hfill
    	\begin{subfigure}[tb]{.49\textwidth}
			\centering
			\includegraphics[width=\textwidth]{../images/5.PNG}
			\caption{}
			\label{fig:0.05-5}
		\end{subfigure}
     	\caption{Спектры лазера ОКГ-13 при различных частотах модуляции периметра. Масштабный фактор 0.05.}
     	\label{fig:0.05}
	\end{figure}

	Лэмбовский провал пронаблюдать не смогли по техническим причинам.

	\Subsection{НАБЛЮДЕНИЕ ВЛИЯНИЯ ОБРАТНОЙ РЕАКЦИИ НА СПЕКТР ИЗЛУЧЕНИЯ HE-NE ЛАЗЕРА}

	Установили зеркало на расстоянии в половину длины резонатора лазера ЛГН 223-1, чтобы отражённый луч возвращался обратно в лазер. В результате небольшой перестройки зеркала получали положительную и отрицательную обратную связь --- моды одной четности усиливались, а моды другой четности ослаблялись почти до нуля.

	Такой эффект удаётся объяснить следующим образом. Зеркало $\text{З}_7$ начинает образовывать с правым зеркалом лазера ЛГН 223-1 резонатор. Его эффективная длина составляет $3/2 \cdot L$. Поэтому, если в резонаторе длины $L$ собственные моды имеют частоты $n \cdot c/2L,\,n\in\mathbb{Z}$, а в резонаторе длины $3/2\cdot L$ собственные моды имеют частоты $2/3\cdot m \cdot c/2L,\,m\in\mathbb{Z}$. Собственные моды таких резонаторов пересекаются на модах с номерами $2n+1$, модуляцию интенсивности которых мы и наблюдали.

	\Subsection{НАБЛЮДЕНИЕ ХАРАКТЕРА СПЕКТРАЛЬНОГО ОТКЛИКА СОБСТВЕННЫХ МОД КОЛЬЦЕВОГО РЕЗОНАТОРА В ЗАВИСИМОСТИ ОТ СОСТОЯНИЯ ПОЛЯРИЗАЦИИ ИСХОДНОЙ ВОЛНЫ}

	Были получены картины двух типов спектров лазера ОКГ-13, соответствующих разным поляризациям (см. рис. \ref{fig:0.2}, \ref{fig:0.05}). С помощью поляризатора было установлено, что центральный пик отвечает полосе излучения, которое поляризовано в плоскости трехзеркального резонатора. Боковые пики в свою очередь поляризованы ортогонально такой плоскости.

	По графикам на осциллографе можно видеть, что моды ортогональных поляризаций расположены на равных расстояниях попеременно, что объясняется тем, что для трёхзеркального кольцевого резонатора спектры собственных мод для поляризаций в плоскости луча и поперёк этой плоскости сдвинуты на половину длины волны, так как при отражении от зеркала TE-волны её фаза меняется на $\pi$, а TM-волны не меняется.

	Как видно из рисунков \ref{fig:0.2}, \ref{fig:0.05} больший максимальный отклик имеет поляризации поперек плоскости трехзеркального резонатора.

	По пяти состояниям поляризации была построена зависимость нормированных значений интенсивности излучения с определенной поляризацией от другой (см. рис. \ref{fig:I-p-vs-I-s}). Получившаяся зависимость достаточно близка к $I_s/\max(I_s) + I_p/\max(I_p) = 1$, что объяснятся тем, что поляризационные управители не поглощают излучение и полная мощность излучения должна в сумме двух поляризаций сохраняться (однако в реальности картина немного другая из-за наличия потерь). 

	\begin{figure}[htbp]
		\centering
		\includegraphics[width=0.7\textwidth]{../images/I-p-vs-I-s.png}
		\caption{Зависимость нормированных значений интенсивности излучения одной поляризации от нормированных значений интенсивности излучения другой поляризации.}
		\label{fig:I-p-vs-I-s}
	\end{figure}

	Измерили ширину резонансных пиков на полувысоте (см. рис. \ref{fig:res}). Мы знали ширину межмодового интервала $c/2L = 666\,\text{МГц}$ в делениях при напряжении $U=1.581\,\text{В}$. Пересчитали её в деления для напряжения $U=0.618\,\text{В}$, что позволило измеренную в делениях ширину пиков пересчитать в герцы. Получили, что $\Delta\nu_s = 22.2\,\text{МГц}$ и $\Delta\nu_p = 34.8\,\text{МГц}$.

	\begin{figure} 
    	\centering
    	\begin{subfigure}[tb]{.49\textwidth}
			\centering
		 	\includegraphics[width=\textwidth]{../images/6.PNG}
			\caption{}
			\label{fig:res-6}
		\end{subfigure}
    	\hfill
    	\begin{subfigure}[tb]{.49\textwidth}
			\centering
			\includegraphics[width=\textwidth]{../images/7.PNG}
			\caption{}
			\label{fig:res-7}
		\end{subfigure}
     	\caption{(a): Резонансный пик, поляризованный поперек плоскости кольцевого резонатора. (b): Резонансный пик, поляризованный в плоскости кольцевого резонатора. Напряжение $U=0.618\,\text{В}$.}
     	\label{fig:res}
	\end{figure}

	Отсюда оценили коэффициент отражения зеркал кольцевого резонатора по формуле $\Delta \nu = c \ln(R^{-3/2})/2L$. Получили, что $R_s = 0.97$ и $R_p = 0.95$.


\end{document}